% gitlog.tex
% Copyright 2015 Brent Longborough
%
% This work may be distributed and/or modified under the
% conditions of the LaTeX Project Public License, either version 1.3
% of this license or (at your option) any later version.
% The latest version of this license is in
%   http://www.latex-project.org/lppl.txt
% and version 1.3 or later is part of all distributions of LaTeX
% version 2005/12/01 or later.
%
% This work has the LPPL maintenance status `maintained'.
% The Current Maintainer of this work is Brent Longborough.
%
% This work consists of these files:
% -----------------------------------------------------
\documentclass[a4paper,12pt,twoside,openany]{memoir}
% =====================================================
\usepackage[british]{babel}
\selectlanguage{british}
\usepackage[style=iso]{datetime2}
\usepackage[local,pcount,grumpy,markifdirty]{gitinfo2}
\usepackage{tgpagella}
\usepackage{tgadventor}
\usepackage{fontspec}
\setmainfont[Numbers={Proportional,OldStyle},Ligatures=TeX]{TeX Gyre Pagella}
\setsansfont[Numbers={Proportional,OldStyle},Ligatures=TeX]{TeX Gyre Adventor}
\setmonofont{Consolas}
\usepackage{enumitem}
\setlist[description]{%
    format=\ttfamily\bfseries,
    style=nextline,
    leftmargin=3em,
    itemsep=0.5\onelineskip}
\setulmarginsandblock{0.11111\paperwidth}{0.22222\paperwidth}{*}
\setlrmarginsandblock{0.11111\paperwidth}{0.22222\paperwidth}{*}
\setheadfoot{1.2\baselineskip}{0.0849\paperwidth}
\setmarginnotes{0.125\foremargin}{0.75\foremargin}{\onelineskip}
\setheaderspaces{*}{*}{0.618}
\checkandfixthelayout[fixed]
% \makepagenote
% \continuousnotenums
% \notepageref
% \foottopagenote
% \renewcommand*{\printpageinnotes}[1]{%
%   (p.\pageref{#1})\space}
% \renewcommand\printpageinnoteshyperref[1]{%
%   (p.\pageref*{#1})\space}
% \renewcommand*{\pagenotesubhead}[3]{%
%   \subsubsection*{#1: #3}}
\tightlists
\chapterstyle{bringhurst}
\pagestyle{empty}
\aliaspagestyle{chapter}{empty}
\settocdepth{subsection}
\setsecnumdepth{none}
\newcommand{\bpara}[1]{\par\vspace{\beforeparaskip}\noindent\textbf{#1}\,}
\newcommand{\rpara}[1]{\par\noindent\textbf{#1}\,}
\newcommand{\dark}[1]{\texttt\textbf{{#1}}}
\newcommand{\sfit}[1]{\textit{#1}}
\newcommand{\git}{\sfit{git}}
\newcommand*{\emailat}{@}
\newcommand{\tpname}{\sfit{gitlog}}
\newcommand{\tpfname}{\textsf{gitlog.sty}}
% -----------------------------------------------------
\usepackage[%
	bookmarksnumbered,
	bookmarksopen,
	linktocpage,
	]{hyperref}
\hypersetup{
   pdfauthor={Brent Longborough},
   pdftitle={The gitlog package: git change logs for LaTeX},
   pdfkeywords={git;changelog;dvcs},
}
\usepackage[%
    write,
	bibfile=gitlog.sample.bib,
	github=Hightor/gitlog,
	title={Sample Git Change Log},
]{gitlog}
\begin{document}
\frontmatter
% -----------------------------------------------------
\title{%
	~\\[2\baselineskip]
	\Huge \tpfname\\[2ex]%
	\Large A proof of concept for automatic typesetting \\of change logs from the \git\ \textsc{dvcs}
	}
\author{Brent Longborough}
\date{\DTMenglishmonthname{\DTMfetchmonth{gitdate}} \DTMfetchyear{gitdate}}
\maketitle

{\centering
Release:\gitRels\ (\gitAbbrevHash)\\
}
% -----------------------------------------------------
\thispagestyle{empty}
\aliaspagestyle{chapter}{plain}
\clearforchapter
\tableofcontents*
% -----------------------------------------------------
\mainmatter
\pagestyle{giruled}
\aliaspagestyle{chapter}{giplain}
\chapter{Introduction}
The \git{} distributed version control system
maintains an historical log of update activity.
The \tpname{} package provides a way automatically 
to typeset such a log, optionally linking commits 
from the typeset log
to one of the online \textsc{dvcs} hosting services.

% - - - - - - - - - - - - - - - - - - - - - - - - - - -
\section{Limitations}

The current release (\gitRel) 
is intended only as a \emph{proof-of-concept}, 
and should not be used for production-level work
unless you're happy with these limitations:
\begin{itemize}
\item Incorrectly-coded documents, when formatted with the
required \sfit{--shell-escape}, 
\textbf{\emph{can cause arbitrary files to be overwritten.}} 
Although this problem is easy to avoid,
there are no built-in protections.
\item The \git\ change log is built and formatted using the 
facilities of \sfit{biblatex} and \sfit{biber}.
The way this is currently implemented makes it unlikely that 
documents using \tpname\ can contain `normal' bibliographies.
\item New lines in the \git\ change log commit messages are 
simply converted to spaces. 
The result is pretty ugly, 
unless you've had the foresight 
to punctuate your commit messages nicely.
\end{itemize}

That said, I think \tpname\ will still be useful to a subset of \TeX\ users,
and I welcome suggestions and contributed code via Github 
\url(https://github.com/Hightor/gitinfo2/issues)
or email.

% - - - - - - - - - - - - - - - - - - - - - - - - - - -
\section{How \tpname\ works}
\begin{enumerate}
\item Whenever you commit work or check out a branch in \git,
\git\ executes a \textit{post-commit} or \textit{post-checkout hook}.

\item The \tpname\ package includes a sample hook
(placed in your \git\ hooks directory),
which extracts metadata from \git\ and writes it to a \TeX\ file,

\item When you format your document, \tpname\ reads
in a series of \LaTeX\ commands.

\item You may use these commands to insert
the metadata you need at any point in the document.
\end{enumerate}

It is important to note that \tpname\ reads the metadata
in the repository (module or submodule) root.

% -----------------------------------------------------
\chapter{Using the package}
\label{ch:using}
To collect and typeset \git{} history,
you load the \tpname\ package in the usual way:\\[0.5\baselineskip]
\texttt{\textbackslash usepackage[$<options>$]\{gitlog\}}

% - - - - - - - - - - - - - - - - - - - - - - - - - - -
\section{Package options}

The following options are available:

\subsection{General options}

\begin{description}

\item[\texttt{title=\textit{log title}}]
This option allows you to change the chapter title associated with 
the typeset change log.
The default is `Change Log'.

\item[\texttt{date}, \texttt{nodate}]
These complementary options
allow you to specify whether or not you want 
the author date and name to be added to the change log.
The default is \texttt{nodate}.


\end{description}

\subsection{Options for controlling the .bib file}

\begin{description}

\item[\texttt{write}, \texttt{nowrite}]
These complementary options
allow you to specify whether or not you want 
\tpname\ to generate the special change log (\sfit{.bib}) 
for you.

If you specify \texttt{write}, then \tpname\ will regenerate the 
change log every time the document is formatted. 
Note that this option is implemented with the \sfit{\\write18} command,
and requires that your document be processed with the \TeX option 
\texttt{--shell-escape}.

If you specify \texttt{nowrite}, then \tpname\ will not write anything,
and the \texttt{--shell-escape} option is not required.
However, in this case, 
you are responsible for generating the change log 
in the correct format for \tpname\ to use, 
as well as being able to hand-tailor it.

If neither option is specified, the default depends on whether or not
you use the \sfit{bibfile} option, described next.  

\clearpage
\item[\texttt{bibfile=\textit{filename}}]
The \git\ change log data is kept in 
a file in \sfit{biblatex .bib} format, 
which \tpname\ writes (if requested) and then reads 
to format the change log.

If this option is not specified, the default filename,
\textit{<jobname>}\texttt{.gitlog.bib}, is used.
In this case, the default option \sfit{write} is used,
but can be suppressed by specifying \texttt{nowrite}.

This can be overridden by specifying your own choice
of filename using this option.
In this case, the default option \sfit{nowrite} is used,
but you can force \tpname\ to write to your file 
by specifying \texttt{write}.

\dark{Warning:} there is \emph{no} protection against writing
to any file name whatsoever. 
The default settings are reasonably `safe'; 
where you need to ship the document without its repository,
then  
\begin{quote}
{\ttfamily
[write,bibfile=\jobname.local.bib]
}
\end{quote}
is probably a safe set of options to use.

\end{description}

\subsection{Options for linking to on-line services}

The change log typeset by \tpname\ can include links
connecting each commit to its corresponding page
in either the GitHub or the Atlassian Bitbucket 
online repository services. 
(Trade marks of their respective owners.)

To use this feature, you must load the \sfit{hyperref} package
before loading \tpname, and use one of the following, 
mutually exclusive, options.

\begin{description}

\item[\texttt{github=\textit{repository-path}}]
When each commit in the changelog is typeset,
a link is generated to the corresponding page on GitHub,
with a \textsc{url} in this format:
\begin{quote}
{\ttfamily\small
https://github.com/{\rmfamily\itshape repository-path}/commit/{\rmfamily\itshape commit-hash}
}
\end{quote}

\item[\texttt{bitbucket=\textit{repository-path}}]
When each commit in the changelog is typeset,
a link is generated to the corresponding page on Atlassian Bitbucket,
with a \textsc{url} in this format:
\begin{quote}
{\ttfamily\small
https://bitbucket.org/{\rmfamily\itshape repository-path}/commits/{\rmfamily\itshape commit-hash}
}
\end{quote}

\end{description}

\clearpage
\section{Typesetting the change log}
% -----------------------------------------------------
\chapter{Etc}
\section{Release notes}

\rpara{R2.0.6: 2015-11-14 -- Detokenise the metadata}
\begin{itemize}
\item Detokenise names, emails, branches, and tags.
This means you should be able to use branch names like \verb!yes@top_dol$lar! without 
\TeX\ chewing you out.
I'd welcome feedback on this, 
as I have an uneasy feeling there may be unintended consequences.
\end{itemize}

\rpara{R2.0.5: 2015-11-09 -- Bug fixes and general improvements}
\begin{itemize}
\item Support for the \sfit{datetime2} package
\item Provide correct committer metadata in hook sample
\item Change Warning to Info when \sfit{gitHeadinfo.gin} is found
\item Move all package dependencies from \sfit{gitexinfo.sty} to \sfit{gitinfo2.sty}
\item Support for \git{} Version 2 log output, with more accurate branch name analysis
\item Only use colours if the \sfit{xcolor} package is loaded
\end{itemize}

\rpara{R2.0.4: 2014-10-03 -- Fixes and documentation improvements}
\begin{itemize}
\item More robust \git\ hooks,
for improved detection of dirty working copies
\item A new section, \textit{\titleref{sect:seqeve}},
to help with doing things in the right order
\item Other minor improvements to the manual
\end{itemize}

\rpara{R2.0.3: 2014-09-05 -- Handle hostile e-mail addresses}
\begin{itemize}
\item Provide an e-mail address wrapper command,
to allow users to tailor protection
against `\_' and other characters in email addresses.
\item This release was not shipped to \textsc{Ctan}
\end{itemize}

\rpara{R2.0.2: 2014-09-04 -- Mostly cosmetic}
\begin{itemize}
\item Fix packaging problems for \textsc{Ctan} and \TeX{}~Live
\item Improve appearance of watermark
\item Improve documentation: correct file references. remove gibberish, extend acknowledgments
\end{itemize}

\clearpage
\section{Acknowledgements}

The \href{http://tex.stackexchange.com}{\TeX.SE community}
has been a constant source of help, inspiration, and amazement.
In particular, I'd like to thank
\href{http://tex.stackexchange.com/users/73/joseph-wright}{Joseph Wright},
who rescued me from the jaws of the TeX parser by explaining
\textbackslash expandafter.

I'd also like to register my thanks to the owners of the packages on which
\tpname\ depends: datetime2, eso-pic, etoolbox, kvoptions, and xstring.

Many people have written to me kindly
I owe you all an apology for the amount of time that elapsed
from your suggestions to the making of \tpname.

In some cases, I have not taken up suggestions other than as food for thought,
in others used the code or suggestions directly, and,
in yet others, adapted.
I thank you all, especially for stimulating my thought processes,
and thus, hopefully,

I think I owe a special mention, both for ideas and code,
to Clea Rees, Jörg Weber, and Kai Mindermann
for improving the handling of \git\ references;
to Jörg Weber for watermarking;
to Michael Rans and Ross Vandegrift for
and to \sfit{ivokabadshow} on GitHub 
for a welcome example of how to detokenise the metadata.

My sincere thanks, too, to
Adrian Burd,
cedb12 (GitHub),
Maximilian Held,
Johannes Hoetzer,
Mikko Korpela,
Martin W Leidig,
Enrico Malizia,
Ken Mankoff,
Ryan Matlock,
Robbie Morrison,
Nik (gwdg nokta de),
Omid (gmail nokta com),
Sasaki~Suguru,
Tor\-bjørn~T (GitHub and TeX.SE),
and
Felix Wenger.

Special thanks to Karl Berry for helping me to
reduce my incompetence with \texttt{ctanify}.
And, of course, for \TeX{} Live and everything else.

Finally, but by no means least,
my thanks to the \textsc{Ctan} elves, and their dæmons,
particularly, in my case,
Ina Dau,
Manfred Lotz, 
Petra Rübe-Pugliese,
and 
Robin Fairbairns, 
for their infinite patience and unstinting
dedication to the \TeX\ community.

The failings, of course, I claim for myself.

% - - - - - - - - - - - - - - - - - - - - - - - - - - -
\clearpage
\section{Copyright \& licence}
Copyright \copyright\ \DTMfetchyear{gitdate}, Brent Longborough,
who has asserted his moral right
to be identified as the author of this work.

This work --- \tpname\ --- may be distributed and/or modified under the
conditions of the LaTeX Project Public License: either version 1.3
of this license, or (at your option) any later version.

The latest version of this license can be found
at the \LaTeX\ Project website,%
\footnote{(\url{http://www.latex-project.org/lppl.txt})}
and version 1.3 or later is part of all recent distributions of
\LaTeX.

This work has the LPPL maintenance status `maintained';
the Current Maintainer of this work is Brent Longborough.

This work consists of the files
gitinfo2.sty, gitexinfo.sty, gitinfo2.tex, gitinfo2.pdf,
gitinfotest.tex, post-xxx-sample.txt,
and gitPseudoHeadInfo.gin.

% - - - - - - - - - - - - - - - - - - - - - - - - - - -
\section{From the author}
Although my limitations as a \TeX nician
mean that I've implemented \tpname\ in a rather simplistic way
that needs some setup that is more complicated than I wanted,
I hope you find the package useful.
I'll be very happy to receive your comments by email.\\[\baselineskip]
Brent Longborough\\[\baselineskip]
\textsf{brent+ctancontrib (bei) longborough (punkt) org}\\
and at \href{http://tex.stackexchange.com/users/344/brent-longborough}{\TeX.SE}
% -----------------------------------------------------
\printGitLog
% -----------------------------------------------------
\clearpage
\raggedright
\printpagenotes
\end{document}
