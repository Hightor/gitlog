% gitlog.tex
% Copyright 2015 Brent Longborough
%
% This work may be distributed and/or modified under the
% conditions of the LaTeX Project Public License, either version 1.3
% of this license or (at your option) any later version.
% The latest version of this license is in
%   http://www.latex-project.org/lppl.txt
% and version 1.3 or later is part of all distributions of LaTeX
% version 2005/12/01 or later.
%
% This work has the LPPL maintenance status `maintained'.
% The Current Maintainer of this work is Brent Longborough.
%
% This work consists of these files:
% -----------------------------------------------------
\documentclass[a4paper,12pt,twoside,openany]{memoir}
% =====================================================
\usepackage[british]{babel}
\selectlanguage{british}
\usepackage[style=iso]{datetime2}
\usepackage[local,pcount,grumpy,markifdirty]{gitinfo2}
\usepackage{tgpagella}
\usepackage{tgadventor}
\usepackage{fontspec}
\setmainfont[Numbers={Proportional,OldStyle},Ligatures=TeX]{TeX Gyre Pagella}
\setsansfont[Numbers={Proportional,OldStyle},Ligatures=TeX]{TeX Gyre Adventor}
\setmonofont{Consolas}
\usepackage{enumitem}
\setlist[description]{%
    format=\ttfamily\bfseries,
    style=nextline,
    leftmargin=5em,
    itemsep=0.5\onelineskip}
\setulmarginsandblock{0.11111\paperwidth}{0.22222\paperwidth}{*}
\setlrmarginsandblock{0.11111\paperwidth}{0.22222\paperwidth}{*}
\setheadfoot{1.2\baselineskip}{0.0849\paperwidth}
\setmarginnotes{0.125\foremargin}{0.75\foremargin}{\onelineskip}
\setheaderspaces{*}{*}{0.618}
\checkandfixthelayout[fixed]
\makepagenote
\continuousnotenums
\notepageref
\foottopagenote
\renewcommand*{\printpageinnotes}[1]{%
  (p.\pageref{#1})\space}
\renewcommand\printpageinnoteshyperref[1]{%
  (p.\pageref*{#1})\space}
\renewcommand*{\pagenotesubhead}[3]{%
  \subsubsection*{#1: #3}}
\tightlists
\chapterstyle{bringhurst}
\pagestyle{empty}
\aliaspagestyle{chapter}{empty}
\settocdepth{subsection}
\setsecnumdepth{none}
\newcommand{\bpara}[1]{\par\vspace{\beforeparaskip}\noindent\textbf{#1}\,}
\newcommand{\rpara}[1]{\par\noindent\textbf{#1}\,}
\newcommand{\dark}[1]{\texttt\textbf{{#1}}}
\newcommand{\sfit}[1]{\textit{#1}}
\newcommand{\git}{\sfit{git}}
\newcommand*{\emailat}{@}
\newcommand{\tpname}{\sfit{gitlog}}
\newcommand{\tpfname}{\textsf{gitlog.sty}}
% -----------------------------------------------------
\usepackage[%
	bookmarksnumbered,
	bookmarksopen,
	linktocpage,
	]{hyperref}
\hypersetup{
   pdfauthor={Brent Longborough},
   pdftitle={The gitlog package: git change logs for LaTeX},
   pdfkeywords={git;changelog;dvcs},
}
\usepackage[%
	bibfile=gitlog.sample.bib,
	github=Hightor/gitlog,
	title={Sample Git Change Log},
]{gitlog}
\begin{document}
\frontmatter
% -----------------------------------------------------
\title{%
	~\\[2\baselineskip]
	\Huge \tpfname\\[2ex]%
	\Large A proof of concept for automatic typesetting \\of change logs from the \git\ \textsc{dvcs}
	}
\author{Brent Longborough}
\date{\DTMenglishmonthname{\DTMfetchmonth{gitdate}} \DTMfetchyear{gitdate}}
\maketitle

{\centering
Release:\gitReln\ (\gitAbbrevHash)\\
}
% -----------------------------------------------------
\thispagestyle{empty}
\aliaspagestyle{chapter}{plain}
\clearforchapter
\tableofcontents*
% -----------------------------------------------------
\mainmatter
\pagestyle{giruled}
\aliaspagestyle{chapter}{giplain}
\chapter{Introduction and Warning}
The \git{} distributed version control system
maintains an historical log of update activity.
The \tpname{} package provides a way automatically to typeset such a log.


More and more, writers are using version control systems
to manage the progress of their works.
One popular distributed version control system commonly used today
is \git.

Among other blessings, \git provides
some useful metadata concerning the history of the developers'
work, and, in particular, about the current state of that work.

\tpname\ allows writers to incorporate some of this metadata
into their documents, to show from which point in their development
a given formatted copy was produced.

% - - - - - - - - - - - - - - - - - - - - - - - - - - -
\section{How \tpname\ works}
\begin{enumerate}
\item Whenever you commit work or check out a branch in \git,
\git\ executes a \textit{post-commit} or \textit{post-checkout hook}.

\item The \tpname\ package includes a sample hook
(placed in your \git\ hooks directory),
which extracts metadata from \git\ and writes it to a \TeX\ file,

\item When you format your document, \tpname\ reads
in a series of \LaTeX\ commands.

\item You may use these commands to insert
the metadata you need at any point in the document.
\end{enumerate}

It is important to note that \tpname\ reads the metadata
in the repository (module or submodule) root.

% -----------------------------------------------------
\chapter{Using the package}
\label{ch:using}
To collect and typeset \git{} history,
you load the \tpname\ package in the usual way:\\[0.5\baselineskip]
\texttt{\textbackslash usepackage[$<options>$]\{gitlog\}}

% - - - - - - - - - - - - - - - - - - - - - - - - - - -
\section{Package options}

The following options are available:

\subsection{General options}

\begin{description}

\item[grumpy]
it will set all the metadata to a common value, ``(None)'',
issue a package warning, and carry on.
If the \texttt{grumpy} option is used,
this warning becomes an error, and processing stops.

\item[\texttt{missing=\textit{text}, notags=\textit{text}, dirty=\textit{text}}]
These three options allow you to tailor the default text used by \tpname\ when
metadata is missing `(None)',
the branch head has no tags `(None)',
and the working copy has uncommitted changes `(*)'.
For example:

\begin{quote}
{\ttfamily
[missing=Help!,notags=\{No\,tags?\},dirty=Eww!]
}
\end{quote}
If you have complex needs, as in the second example,
don't forget to enclose your text in \{\}s.

\item[\texttt{maxdepth=\textit{n}}]
\tpname\ starts in the directory containing the master document,
and searches up the directory tree until it finds it.
This search is limited to {\ttfamily\itshape n} levels --- 4 by default.
If have to deal with documents
deeper in your repository tree, you can extend this limit with, say,
{\ttfamily maxdepth=8}.

\end{description}
\clearpage
\subsection{Options for watermarking}

These options allow you to place a watermark of \git\ metadata,
at the bottom of the paper, conditionally or unconditionally

\begin{description}

\item[mark]
This option causes \tpname\ to generate a watermark,
centred at the bottom of each sheet,
containing `useful' \git\ metadata.
The watermark is always added.

\item[markifdraft]
This option is like {\ttfamily\bfseries mark,}
but only activates the watermark if the document is being processed
with the {\ttfamily\bfseries draft} option.

\item[markifdirty]
This option is like {\ttfamily\bfseries mark,}
but only activates the watermark if the last commit or checkout left
uncommitted changes in the repository.

\item[marknotags]
If \tpname\ can't find any tags in the \git\ references,
it suppresses the second line of the watermark.
If, however, you would like the second line always to appear,
add this option to the package options.

\item[raisemark=\textit{vertical space}]
By default, \tpname\ sets the bottom line of the watermark
at 1.5 baseline
skips\footnote{\texttt{1.5\textbackslash{}baselineskip},
an admittedly arbitrary value, chosen for my Canon printer.}
above the bottom of the paper.
If you prefer something different,
you can specify it here.
For example:

\begin{quote}
{\ttfamily
[raisemark=0.95\textbackslash{}paperheight]
}
\end{quote}


\item[draft]
This option \emph{should not be used;} it only exists to `capture' the
{\ttfamily\bfseries draft} option from the document class definition.

\end{description}

\subsection{Options for \sfit{memoir} users}

For more about these options,
please read
%The use of these options is described in more detail in
`\titleref{sect:memuser}'
on page~\pageref*{sect:memuser}.

\begin{description}

\item[footinfo]
This option is no longer used, and if present is silently ignored.
For \sfit{memoir} users, \tpname\ now creates, automatically,
three new page styles.

\item[pcount]
For \sfit{memoir} users, this option will replace the folio
in the new page styles with one of the form \textit{x/y},
where \textit{x} is the folio and \textit{y} is the page count.

No warning is given, and no action taken,
if this parameter is used with another document class.

\end{description}
% - - - - - - - - - - - - - - - - - - - - - - - - - - -
\clearpage
\section{The metadata}
The \git\ metadata, for the current HEAD commit,
is made available in the document
as a series of parameter-less \LaTeX commands.
Here they are:

\begin{description}

\item[gitReferences]
    A list of any \git\ references (tags, branches) associated
    with this commit.
    This string is not for the faint-hearted;
    its format and order may vary between versions of \git.\\
    Example:\\\textit{\small\gitReferences}

\item[gitBranch]
    The name of the current branch.
    Depending on how you use \git,\footnote{For example, checking out unnamed branches}
    this information may not be available,
    and will then be shown as the default or specified value of
    the \texttt{missing} package option.
    For \git{} versions before 2.0.0, where the current HEAD commit refers
    to more than one branch head, this value may be different from the
    currectly checked-out branch.\\
    Example: \textit{\gitBranch}

\item[gitDirty]
    If the last commit or checkout left uncommitted changes in the working tree,
    the default or specified value of the \texttt{dirty} package option;
    otherwise empty.

\item[gitAbbrevHash]
    The seven-hex-char abbreviated commit hash\\
    Example: \textit{\gitAbbrevHash}

\item[gitHash]
    The full 40-hex-character commit hash\\
    Example: \textit{\gitHash}

\item[gitAuthorName]
    The name of the author of this commit\\
    Example: \textit{\gitAuthorName}

\item[gitAuthorEmail]
    The email address of the author of this commit\\
    Example: \textit{myemail\emailat evilspam.net}

\item[gitAuthorDate]
    The date this change was committed by the author,
    in the format \textit{yyyy-mm-dd}\\
    Example: \textit{\gitAuthorDate}

\item[gitAuthorIsoDate]
    The date and time this change was committed by the author,
    in ISO format\\
    Example: \textit{\gitAuthorIsoDate}
\clearpage
\item[gitAuthorUnixDate]
    The date and time this change was committed by the author,
    as a Unix timestamp\\
    Example: \textit{\gitAuthorUnixDate}

\item[gitCommitterName]
    The name of the committer of this commit\\
    Example: \textit{\gitCommitterName}

\item[gitCommitterEmail]
    The email address of the committer of this commit\\
    Example: \textit{watcher\emailat gchq.gov.uk}

\item[gitCommitterDate]
    The date this change was committed by the committer,
    in the format \textit{yyyy-mm-dd}\\
    Example: \textit{\gitCommitterDate}

\item[gitCommitterIsoDate]
    The date and time this change was committed by the committer,
    in ISO format\\
    Example: \textit{\gitCommitterIsoDate}

\item[gitCommitterUnixDate]
    The date and time this change was committed by the committer,
    as a Unix timestamp\\
    Example: \textit{\gitCommitterUnixDate}

\end{description}


\subsection{Additional metadata (Version 1)}

Three more commands are available, but their use should be considered
experimental. \tpname\ searches the \git\ references metadata for
anything (probably a \git\ tag) that looks like a number with a decimal point.
The first such number it finds is taken as a ``Version Number''
and made available in three different formats, explained here:

\begin{description}
\item[gitVtag]
    The version number, without decorations. If no version number is found,
    empty (i.e. zero width).
\item[gitVtags]
    The version number, with a leading space. If no version number is found,
    empty.
\item[gitVtagn]
    The version number, with a leading space.
    If no version number is found, a space,
    followed by the default or specified value of
    the \texttt{missing} package option.\\
    Example: \textit{\gitVtagn}
\end{description}

These versioning tags have been superseded by release tags in Version 2,
although they should continue to work as before.

\subsection{Additional metadata (Version 2)}

From Version 2 onwards, additional \git\ metadata is available,
in general improving or extending the facilities available in Version 1.
The Version 1 metadata is retained for backward compatibility.

Included is a new set of \texttt{gitRel} commands,
designed to replace \texttt{gitVtag} and its cousins.
\tpname\ searches the \git\ metadata for tags
on the current \textsc{head} commit or its ancestors,
and makes the first tag found available.
It also looks for the latest such tag
whose name begins with a digit, and which contains a full stop (period),
and makes the tag, and the number of commits following it,
available as a `release number'.%
\footnote{\tpname\ doesn't check for numerics, so you can use a tag like `1.5-beta' if you wish.}
Here are the new commands in Version~2:

\begin{description}

\item[gitFirstTagDescribe]
	The last tag reachable from the current \textsc{head}.
    Please see git-describe for more information.
    If the working copy is \textit{dirty} (has uncommitted changes),
    the string has '-*' appended.\\
    Example: \textit{\gitFirstTagDescribe}

\item[gitRel]
    The release number, without any decorations.
    If no release number is found, empty (i.e. zero width).

\item[gitRels]
    The release number, with a leading space.
    If no release number is found, empty.

\item[gitReln]
    The release number, with a leading space.
    If no release number is found, a space,
    followed by the default or specified value of
    the \texttt{missing} package option.

\item[gitRoff]
    The number of commits between the current \textsc{head}
    and the tag holding the release number.
    If the tag refers to the current \textsc{head}, zero.

\item[gitTags]
    A comma-separated list of tags associated with the current \textsc{head}.\\
    Example: \textit{\gitTags}

\item[gitDescribe]
    The raw output from the git-describe command for the last release tag
    reachable from the current \textsc{head}, including tag name, commit offset,
    short hash, and a dirty flag.
    Please see git-describe for more information.
    Example: \textit{\gitDescribe}

\end{description}

\clearpage
\subsection{Watermark tailoring commands}

If you use the Version 2 options to place a watermark,
you can tailor the format of the watermark to suit your needs,
by redefining one or more of the following commands.

\begin{description}

\item[gitMark]
    Contains the text of the watermark.
    Output from the default definition can be seen
    below the footer
    at the bottom of this page.
    The definition
    (here split onto four lines to fit)
    is:

\begin{verbatim}
Branch: \gitBranch\,@\,\gitAbbrevHash{}
\textbullet{}
Release:\gitReln{} (\gitAuthorDate)\\
Head tags: \gitTags
\end{verbatim}
You can tailor this by redefining \verb!\gitMark!.
For example:

\begin{verbatim}
\renewcommand{\gitMark}{\gitHash\hfill\gitRel}
\end{verbatim}

\item[gitMarkFormat]
    Defines typesetting parameters for the whole watermark.
    The default definition is:

\begin{verbatim}
\color{gray}\small\sffamily
\end{verbatim}
if the \sfit{xcolor} package is loaded; if not, \verb!! is simply omitted.
You can tailor this by redefining \verb!\gitMarkFormat!. For example:

\begin{verbatim}
\renewcommand{\gitMarkFormat}{\color(red)\ttfamily}
\end{verbatim}

\item[gitMarkPref]
    Contains the text of the watermark prefix,
    which depends on the reason the document is being watermarked.
    The default values are \sfit{[Dirty]}, \sfit{[Draft]}, or \sfit{[git]}.
    You can tailor this by redefining \verb!\gitMarkPref!. For example:

\begin{verbatim}
\renewcommand{\gitMarkPref}{[Pending review]}
\end{verbatim}

\end{description}

% - - - - - - - - - - - - - - - - - - - - - - - - - - -
\clearpage
\section{Handling \TeX-hostile e-mail addresses}
\label{sect:emailuser}
Occasionally, a repository will contain `\TeX-hostile'
e-mail addresses such as \verb!my_name@somewhere.net!.
As a result, using the \dark{gitAuthorEmail}
or \dark{gitCommitterEmail} commands can cause errors.

\tpname\ provides an e-mail wrapping command,
\verb!\gitWrapEmail!, to allow you to tailor your use
of email addresses. Its default definition does nothing:
\begin{verbatim}
\newcommand{\gitWrapEmail}[1]{#1}
\end{verbatim}
You can tailor this by redefining \verb!\gitWrapEmail!.
For example, a number of packages
(\sfit{hyperref} is one) provide a \verb!\url! command
which provides the necessary protection.
Using such a package (independently of \tpname),
you can redefine the \tpname\ wrapper in this way:
\begin{verbatim}
\renewcommand{\gitWrapEmail}[1]{\url{#1}}
\end{verbatim}

Please note that from Release 2.0.6 this precaution is optional,
since \tpname\ now detokenises Author and Committer names and email addresses,
and all \git\ reference names.

% - - - - - - - - - - - - - - - - - - - - - - - - - - -
\section{For \sfit{memoir} users}
\label{sect:memuser}
If you use \sfit{memoir}, \tpname\ provides you with
three new pagestyles, based on
\sfit{plain}, \sfit{ruled}, and \sfit{headings}.
The new pagestyles are called
\sfit{giplain}, \sfit{giruled}, and \sfit{giheadings}.

For the \sfit{giplain} and \sfit{giruled} pagestyles,
the folio is moved from the centre to the outer margin of the footer,
and a revision stamp is placed in the inner margin.

For the \sfit{giheadings} pagestyle,
the folio is moved from the outer margin of the header
to the outer margin of the footer,
and a revision stamp is placed in the inner margin of the footer.

If you use the \texttt{pcount} option, a solidus, and the page count,
are appended to the folio.

The revision stamp is generated by this fragment:

\begin{verbatim}
    Release\gitRels: \gitAbbrevHash{} (\gitAuthorDate)
\end{verbatim}
\noindent
which is set at tiny in the sans-serif font.

Note that, in contrast to version 1 of \tpname,
version 2 no longer modifies the existing page styles.
If you wish to use this facility,
you must now select
the appropriate \sfit{gi\textellipsis} page style
explicitly.

You can see an example in the footer of this page,
above the \tpname\ watermark.

% - - - - - - - - - - - - - - - - - - - - - - - - - - -
\section{For \sfit{datetime2} users}
\label{sect:dtuser}
If you use Nicola Talbot's excellent \sfit{datetime2} package,
\tpname\ copies the value of the tag {\ttfamily\bfseries gitAuthorDate}
as a new date object named \sfit{gitdate},
which you can then refer to with other \sfit{datetime2}
functions.

Please refer to the \sfit{datetime2} manual for
further details,
especially in the section entitled `Saving Dates'.

If you use it, you must load \sfit{datetime2}
before you load \tpname.

The \sfit{datetime} package is also supported, 
though deprecated. 

% - - - - - - - - - - - - - - - - - - - - - - - - - - -
\section{Notes on the sequence of events}
\label{sect:seqeve}
Users of \tpname\ (including me) are frequently surprised
by what appear to be incorrect results in their output,%
\footnote{Most frequent are `wrong tag' and `falsely flagged as dirty'}
but are determined in fact by the precise sequence of operations
on a working copy.

As it's impossible to foresee every possible workflow,
what follows is a sequence of steps of a simple workflow
which will serve as an example to help users
to understand what is happening, and to avoid it.

The example repository manages two main files:
\sfit{abc.tex} and its output file, \sfit{abc.pdf}.%
\footnote{Although derived files are often excluded from version control,
the .pdf file in this case might need to be archived ---
think of versions of a proposal, for example.}
Intuitively, the overall sequence of one update \& release cycle
might be like this:

\begin{enumerate}
\item Edit \sfit{abc.tex}, (perhaps) commit intermediate changes,
format the output into \sfit{abc.pdf}, and
check and repeat until ready for release;
\item \label{wf:tex} Commit the release version of \sfit{abc.tex};
\item \label{wf:tag} Tag the release;
\item \label{wf:git} (Missing step);
\item \label{wf:pdf} Format the release version of \sfit{abc.pdf}; and
\item Commit the release version of \sfit{abc.pdf}.
\end{enumerate}

\noindent For reasons that I hope will become obvious,
that doesn't work.
The \git\ metadata is extracted and stored at step (\ref{wf:tex}).
At that point, the release tag doesn't yet exist,
and the working copy is dirty,
since \sfit{abc.pdf} has been regenerated, and has changed
since the last recorded version.

It's simple to fix; all we have to do is
ensure the working copy is clean and regenerate
the saved metadata (now with correct tag information).
Here's the missing step, kept, perhaps, somewhere in a \sfit{makefile:}

\begin{enumerate}
\item[\ref{wf:git}.] \texttt{git checkout abc.pdf}
\end{enumerate}

\noindent This will revert the .pdf file to its repository version,
and then regenerate the stored metadata,
including the release tag.
Step (\ref{wf:pdf}) will now produce output showing the release tag
and a clean working copy.
% -----------------------------------------------------
\chapter{Etc}
\section{Release notes}

\rpara{R2.0.6: 2015-11-14 -- Detokenise the metadata}
\begin{itemize}
\item Detokenise names, emails, branches, and tags.
This means you should be able to use branch names like \verb!yes@top_dol$lar! without 
\TeX\ chewing you out.
I'd welcome feedback on this, 
as I have an uneasy feeling there may be unintended consequences.
\end{itemize}

\rpara{R2.0.5: 2015-11-09 -- Bug fixes and general improvements}
\begin{itemize}
\item Support for the \sfit{datetime2} package
\item Provide correct committer metadata in hook sample
\item Change Warning to Info when \sfit{gitHeadinfo.gin} is found
\item Move all package dependencies from \sfit{gitexinfo.sty} to \sfit{gitinfo2.sty}
\item Support for \git{} Version 2 log output, with more accurate branch name analysis
\item Only use colours if the \sfit{xcolor} package is loaded
\end{itemize}

\rpara{R2.0.4: 2014-10-03 -- Fixes and documentation improvements}
\begin{itemize}
\item More robust \git\ hooks,
for improved detection of dirty working copies
\item A new section, \textit{\titleref{sect:seqeve}},
to help with doing things in the right order
\item Other minor improvements to the manual
\end{itemize}

\rpara{R2.0.3: 2014-09-05 -- Handle hostile e-mail addresses}
\begin{itemize}
\item Provide an e-mail address wrapper command,
to allow users to tailor protection
against `\_' and other characters in email addresses.
\item This release was not shipped to \textsc{Ctan}
\end{itemize}

\rpara{R2.0.2: 2014-09-04 -- Mostly cosmetic}
\begin{itemize}
\item Fix packaging problems for \textsc{Ctan} and \TeX{}~Live
\item Improve appearance of watermark
\item Improve documentation: correct file references. remove gibberish, extend acknowledgments
\end{itemize}

\clearpage
\section{Acknowledgements}

The \href{http://tex.stackexchange.com}{\TeX.SE community}
has been a constant source of help, inspiration, and amazement.
In particular, I'd like to thank
\href{http://tex.stackexchange.com/users/73/joseph-wright}{Joseph Wright},
who rescued me from the jaws of the TeX parser by explaining
\textbackslash expandafter.

I'd also like to register my thanks to the owners of the packages on which
\tpname\ depends: datetime2, eso-pic, etoolbox, kvoptions, and xstring.

Many people have written to me kindly
I owe you all an apology for the amount of time that elapsed
from your suggestions to the making of \tpname.

In some cases, I have not taken up suggestions other than as food for thought,
in others used the code or suggestions directly, and,
in yet others, adapted.
I thank you all, especially for stimulating my thought processes,
and thus, hopefully,

I think I owe a special mention, both for ideas and code,
to Clea Rees, Jörg Weber, and Kai Mindermann
for improving the handling of \git\ references;
to Jörg Weber for watermarking;
to Michael Rans and Ross Vandegrift for
and to \sfit{ivokabadshow} on GitHub 
for a welcome example of how to detokenise the metadata.

My sincere thanks, too, to
Adrian Burd,
cedb12 (GitHub),
Maximilian Held,
Johannes Hoetzer,
Mikko Korpela,
Martin W Leidig,
Enrico Malizia,
Ken Mankoff,
Ryan Matlock,
Robbie Morrison,
Nik (gwdg nokta de),
Omid (gmail nokta com),
Sasaki~Suguru,
Tor\-bjørn~T (GitHub and TeX.SE),
and
Felix Wenger.

Special thanks to Karl Berry for helping me to
reduce my incompetence with \texttt{ctanify}.
And, of course, for \TeX{} Live and everything else.

Finally, but by no means least,
my thanks to the \textsc{Ctan} elves, and their dæmons,
particularly, in my case,
Ina Dau,
Manfred Lotz, 
Petra Rübe-Pugliese,
and 
Robin Fairbairns, 
for their infinite patience and unstinting
dedication to the \TeX\ community.

The failings, of course, I claim for myself.

% - - - - - - - - - - - - - - - - - - - - - - - - - - -
\clearpage
\section{Copyright \& licence}
Copyright \copyright\ \DTMfetchyear{gitdate}, Brent Longborough,
who has asserted his moral right
to be identified as the author of this work.

This work --- \tpname\ --- may be distributed and/or modified under the
conditions of the LaTeX Project Public License: either version 1.3
of this license, or (at your option) any later version.

The latest version of this license can be found
at the \LaTeX\ Project website,%
\footnote{(\url{http://www.latex-project.org/lppl.txt})}
and version 1.3 or later is part of all recent distributions of
\LaTeX.

This work has the LPPL maintenance status `maintained';
the Current Maintainer of this work is Brent Longborough.

This work consists of the files
gitinfo2.sty, gitexinfo.sty, gitinfo2.tex, gitinfo2.pdf,
gitinfotest.tex, post-xxx-sample.txt,
and gitPseudoHeadInfo.gin.

% - - - - - - - - - - - - - - - - - - - - - - - - - - -
\section{From the author}
Although my limitations as a \TeX nician
mean that I've implemented \tpname\ in a rather simplistic way
that needs some setup that is more complicated than I wanted,
I hope you find the package useful.
I'll be very happy to receive your comments by email.\\[\baselineskip]
Brent Longborough\\[\baselineskip]
\textsf{brent+ctancontrib (bei) longborough (punkt) org}\\
and at \href{http://tex.stackexchange.com/users/344/brent-longborough}{\TeX.SE}
% -----------------------------------------------------
\printGitLog
% -----------------------------------------------------
\clearpage
\raggedright
\printpagenotes
\end{document}
